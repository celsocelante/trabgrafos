\documentclass[a4paper,10pt]{article}
\usepackage{graphicx}
\usepackage[brazil]{babel}
\usepackage[utf8]{inputenc}
\usepackage{amsmath}

\title{Universidade Federal do Espírito Santo\\Departamento de Informática\\Profa Claudia Boeres\\Teoria dos Grafos 2014/2\\ \textbf{Trabalho Computacional}}
\date{}

\pdfinfo{%
  /Title    ()
  /Author   ()
  /Creator  ()
  /Producer ()
  /Subject  ()
  /Keywords ()
}

\begin{document}
\maketitle

\section{Descrição do problema}
 \begin{large}
\textbf{Enviando emails}
\end{large}
\newline
\newline
Considere uma rede composta por $n$ servidores SMTP ligados por $m$
cabos de rede. Cada um dos m cabos conecta dois computadores e tem
uma certa latência medida em milissegundos necessária para enviar
uma mensagem de e-mail. Faça um programa que calcule o menor tempo
necessário para enviar uma mensagem do servidor $S$ ao servidor $T$
por uma sequência de cabos. Assume-se que não há atrasos
incorridos em qualquer um dos servidores. Use o algoritmo de
Dijkstra na resolução desse problema.
\subsection{Entrada de Dados}

A entrada de dados consiste em um arquivo composto de vários
conjuntos de teste, dispostos em sequencia no arquivo e separados
por uma linha em branco. A organização das linhas de cada conjunto
de teste é dada por:

\begin{enumerate}
\item A primeira linha do conjunto de teste contém o número n de
computadores a serem conectados, com $2 \leq n \leq 20000$; o número
$m$ de cabos da rede $(0 \leq m \leq 50000)$; e os rótulos
identificadores de $S (0 \leq S < n)$ e $T (0 \leq T < n)$, $S \neq
 T$;
 
 \item As próximas $m$ linhas do conjunto de teste descrevem
informações de cada cabo de rede. Cada uma dessas linhas
devem conter os números identificadores dos servidores
conectados por um cabo bidirecional (com rótulos
identificadores no intervalo $[0, n-1]$), além do valor da
latência $w$, ao longo do cabo $(0 \leq w \leq 10000)$.

\end{enumerate}

\subsection{Saída de Dados}
O arquivo de saída deve conter cada linha (separada por linhas em
branco) referente à resposta do seu programa para cada conjunto de
teste. Assim, cada linha deve possuir a informação do número de
milissegundos requeridos para enviar uma mensagem de $S$ a $T$. Se há
um caminho entre $S$ e $T$, a mensagem emitida pelo programa deve ser
“Envio de mensagens de $<S>$ a $<T>$: $<tempo em ms>$.”. Caso contrário,
a mensagem deve ser: “Envio de mensagens de $<S>$ a $<T>$: não há
caminho entre $<S>$ e $<T>$.”.

\subsubsection{Exemplo}


\includegraphics[scale=0.5]{graph.png}

\section{ Instâncias de Teste}

Os algoritmos devem ser executados para pelo menos 5
redes de entrada (a primeira delas deve ser referente
aos exemplos dados para cada problema). As outras 4
redes podem ser geradas com valores distintos de $n$ e $m$,
mas respeitando o máximo de $n$ = 150 nós.


\section{Linguagem de Programação}

A escolha da linguagem de programação é livre.

\end{document}
